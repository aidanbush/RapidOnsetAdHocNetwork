\section{Mobile Ad-hoc}
\subsection*{}


\begin{frame}[t]
  \frametitle{Signal Strength Based Link Sensing}
  
  Associativity Based Routing (ABR)
  \begin{itemize}
  \item Connections are determined to be Stable or Unstable
  \irem Uses packet tests to check connection
  \item Changes after detecting lost packets in a row
  \item Prefrence to stable links when selecting route
  \end{itemize}

  \vfill
  
  Protocol Based Signal Strength
  \begin{itemize}
  \item Connections are determined to be strong or weakly connected
  \item Uses RSSI comparison to determin a connections strength
  \item Based on the interval of strong and weak connectivity to determin route
  \end{itemize}

  \vfill

  Weaknesses
  \begin{itemize}
    \item Assume there is an exisiting link
    \item Attempt to avoid link breakage based on signal strength
    \item Use information to sellect a path with stable links
  \end{itemize}

  \vfill

  \begin{flushleft}
    \begin{tiny}
      \begin{minipage}{1.0\linewidth}
        \fullcite{AK03}
      \end{minipage}
    \end{tiny}
  \end{flushleft}
  
\end{frame}

\begin{frame}[t]
  \frametitle{Optimized Link State Routing}

  OLSR
  \begin{itemize}
  \item Uses a partialy fixed network of nodes
  \item Not suited to a mobile ad-hoc network where link breakage is common
  \item Connection is unstable after 2 lost pings
  \item Between missed pings the sent packets are lost
  \end{itemize}

  \vfill

  Signal Strength Based Link in OLSR
  \begin{itemize}
  \item  Predict link breakage before they occure
  \item  Reward improved or strong signal strength
  \item  Punish weakening or weak signal strength 
  \end{itemize}

  \vfill

  \begin{flushleft}
    \begin{tiny}
      \begin{minipage}{1.0\linewidth}
        \fullcite{AK03}
      \end{minipage}
    \end{tiny}
  \end{flushleft}
  
\end{frame}

\begin{frame}[t]
  \frametitle{Signal Strength Based Link Algorith}

  Assumptions
  \begin{itemize}
  \item The network will be set up in an open space
  \item The recived power measurement is delivered by hardware
  \item Always have a fixed node near a mobile node
  \end{itemize}

  \vfill
  
  Algorithm
  \begin{itemize}
  \item Will use two thresholds for implementing RSSI
    \begin{itemize}
      \item[--] A low threshold will be used to indicate an unstable connection
      \item[--] A high threshold will be used to indicate a stable connection
    \end{itemize}
  \item The diffrence between packet signal strengths is evaluated to affect
    the connection
    \begin{itemize}
      \item[--] If the signal strength from the previous packet and current
        packet increase it reward the connection
      \item[--] If the signal strength from the previous packet and current
        packet decreases it punishes the connection
    \end{itemize}
  \end{itemize}

  \vfill

  \begin{flushleft}
    \begin{tiny}
      \begin{minipage}{1.0\linewidth}
        \fullcite{AK03}
      \end{minipage}
    \end{tiny}
  \end{flushleft}

\end{frame}    

\begin{frame}[t]
  \frametitle{Operational Test}

    Used multiple set up methods for the ad-hoc network
    \begin{itemize}
      \item Sequential line
      \item Disperced deployment
    \end{itemize}
    
    After deployment check overall network performace \linebreak
    Use Three test for connectivity of nodes
    \begin{itemize}
    \item Stationary to Stationary
    \item Stationary to Mobile
    \item Mobile to Mobile
    \end{itemize}
    
  \vfill
  
  Conclusions
  \begin{itemize}
  \item Use of both a packet loss and signal strength check made a more robust
    network
  \item Use of algorithm improved performace and helps prevent loops in the
    letwork
  \item Increased capacity utilization
  \end{itemize}
  
  \vfill
  
\end{frame}

\begin{frame}[t]
  \frametitle{Relevance to Project}
  
  RSSI
  \begin{itemize}
  \item Strengths
    \begin{itemize}
      \item[--] Able to adjust the signal strength that node disconnects
    \end{itemize}
  \item Weakness
    \begin{itemize}
      \item[--] Objects interfere with signal strength
    \end{itemize}
  \end{itemize}

  \vfill

  Packet Loss
  \begin{itemize}
  \item Strengths
    \begin{itemize}
      \item[--] Quick to disconnect a bad node
    \end{itemize}
  \item Weakness
    \begin{itemize}
      \item[--] Can lead to less efficent or unstable networks
    \end{itemize}
  \end{itemize}

  \vfill
  
  \begin{flushleft}
    \begin{tiny}
      \begin{minipage}{1.0\linewidth}
        \fullcite{AK03}
      \end{minipage}
    \end{tiny}
  \end{flushleft}

\end{frame}
