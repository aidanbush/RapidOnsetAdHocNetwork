\section{TCP Ad Hoc}
\subsection*{}


\begin{frame}[t]
  \frametitle{Introduction}
  Review of Klemm’s et. al, Improving TCP performance in ad hoc networks using signal strength based link management and Relation to CMPT 464, A2

  Elliott Sobek
  \begin{flushleft}
    \begin{tiny}
      \begin{minipage}{1.0\linewidth}
        \fullcite{klemmet_et_al}
      \end{minipage}
    \end{tiny}
  \end{flushleft}

  \vfill
\end{frame}

\begin{frame}[t]
  \frametitle{Introduction}
  \begin{itemize}
  \item TCP performance in ad hoc networks
  \item Seek to increase TCP performance in network
  \item Proposed solution to implement design uses signal strength
  \item Proposed solution only effective for solving mobility issues
  \item Simulation results are promising
  \item UDP not considered
  \end{itemize}

  \vfill

\end{frame}

\begin{frame}[t]
  \frametitle{Crux}
  \begin{itemize}
  \item Would introducing a connection characteristic such as signal strength improve a mobile TCP ad hoc network’s performance?
  \end{itemize}

  \vfill

\end{frame}

\begin{frame}[t]
  \frametitle{Related Work}
  \begin{itemize}
  \item Use of explicit link failure and explicit rout establishment notifications
  \item Use of fixed RTO
  \item Introduction of the COPAS protocol
  \item Route failure prediction
  \item Pre-emptive routing scheme

  \item No pervious work salvages transit packets
  \item Klemm’s design seeks to solve this
  \end{itemize}

  \vfill

\end{frame}

\begin{frame}[t]
  \frametitle{Packet losses in ad hoc networks}
  \begin{itemize}
  \item Node mobility and link layer congestion are the two main reasons for packet losses
  \item During a situation of congestion false link failures may arise
  \item Transmission range 250m and interference range of 550m
  \end{itemize}

  \vfill

\end{frame}

\begin{frame}[t]
  \frametitle{Reducing Link Failures}
  \begin{itemize}
  \item Mechanism used to reduce packet loss in mobile network is measuring signal strength at the physical layer
  \item First determine if a link failure due to mobility or congestion
  \item Coping with false link failures
  \item Two mechanisms for alleviating the effects of mobility on TCP performance, Proactive and Reactive Link Management
  \item Modification to the AOBV
  \end{itemize}

  \vfill

\end{frame}

\begin{frame}[t]
  \frametitle{Findings/Disscussion}
  \begin{itemize}
  \item Simulation results
  \item Effects of traffic loaf
  \item Effects of node mobility
  \item Higher packet loss due to mobility the better greater the improvement
  \end{itemize}

  \vfill

\end{frame}

\begin{frame}[t]
  \frametitle{Future Work}
  \begin{itemize}
  \item The design of smart techniques to estimate the level of congestion in a network
  \item Correctly determine the levels of congestion of the network
  \end{itemize}

  \vfill

\end{frame}

\begin{frame}[t]
  \frametitle{Relation}
  \begin{itemize}
  \item Use of signal strength to determine positive node loss
  \item Deploying nodes have a relation to signal strength to find optimal deployment location
  \item Power level management improves packet throughput
  \end{itemize}

  \vfill

\end{frame}
