\section{Positioning}
\subsection{}


\begin{frame}[t]
  \frametitle{Introduction}
  Locationing In Distributed Ad-Hoc Wireless Sensor Networks

  Chris Savarese, Jan M. Rabaey
  \begin{itemize}
  \item Berkeley Wireless Research Center
  \end{itemize}

  Jan Beutel
  \begin{itemize}
  \item Computer Engineering and Networks Lab @ ETH Zurich
  \end{itemize}

  Date of Conference: May 7-11, 2001
  Published by IEEE on Aug 7, 2002

  \begin{flushleft}
    \begin{tiny}
      \begin{minipage}{1.0\linewidth}
        \fullcite{AK04}
      \end{minipage}
    \end{tiny}
  \end{flushleft}

  \vfill
\end{frame}

\begin{frame}[t]
  \frametitle{Purpose}
  Why do we want the positions of nodes in an ad hoc network?
  \begin{itemize}
  \item Deployment
  \item Military applications
    \begin{itemize}
    \item GPS is unreliable in some combat scenarios
    \item Faster than GPS since it doesn't require a centralized system
    \item Soldiers can find their position using \textbf{Triangulation}
    \end{itemize}
  \end{itemize}

  \vfill
  
\end{frame}

    

\begin{frame}[t]
  \frametitle{Navigation}
  
  Positioning in Multihop Networks
  \begin{itemize}
  \item Distance Measurements
    \begin{itemize}
    \item RSSI
    \item Angle of arrival (AOA)
    \item Time of arrival (TOA)
    \item Time-distance of arrival (TDOA)
    \end{itemize}
  \item Triangulation
    \begin{itemize}
      \item At least three distance measurements are required to calculate
        Triangulation.
      \item Calculated using a system of linear equations
    \end{itemize}
  \end{itemize}

  \vfill

\end{frame}

\begin{frame}[t]
  \frametitle{Challenges}

  \begin{itemize}
  \item Triangulation
    \begin{itemize}
    \item Anchor Nodes
    \item TDOA requires high synchronization
    \item AOA requires costly antenna arrays
    \end{itemize}
  \item RSSI
    \begin{itemize}
      \item Sensitive to multi-path, interference and non-line of sight
        scenarios
    \end{itemize}
  \end{itemize}

  \vfill

\end{frame}

\begin{frame}[t]
  \frametitle{Triangulation}

  Range measurements are received from a large amount of neighboring anchor
  nodes.

  Least-mean squares approach is used to solve the triangluation problem.

  Results with a 5\% range error was achieved.

  \vfill

\end{frame}

\begin{frame}[t]
  \frametitle{Topology Discovery}

  Topology Discovery is done by using the Assumption Based Coordinates (ABC)
  algorithm.

  \begin{itemize}
  \item Can return relative, or absolute position.
  \item Assumes the first node is located at (0,0,0) or GPS determined origin.
  \end{itemize}
  \begin{center}
    x2 = \frac{r^2_{01} + r^2_{02} + r^2_{12}}{2r_{01}}
  \end{center}
  \begin{center}
    y2 = \sqrt{r^2_{02} - x^2_{2}}
  \end{center}
  \begin{itemize}
  \item Where r is the RSSI determined distance
  \end{itemize}

  \vfill
\end {frame}

\begin{frame}[t]
  \frametitle{Global Positioning}

  Cooperative Ranging is used to converge nodes to a global solution.
  \begin{itemize}
  \item A global resource is needed.
  \item Routing bottlenecks and lots of wasted energy
  \end{itemize}

  Each node concurrently executes functions.
  \begin{itemize}
  \item Receive distance information from neighboring nodes
  \item Solve distance problem
  \item Transmit results to neighboring nodes
  \end{itemize}
  

\begin{frame}[t]
  \frametitle{Implementation}

  In regards to our project, the ABC algorithm can be used for deployment.
  
  You would have to know distance ahead of time.

  Compare network distance to actual distance, deploy if \% error is within a
  threshold.

  \vfill

\end {frame}

  

  
  
    
