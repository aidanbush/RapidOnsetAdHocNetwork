\section{Implementation}
\subsection*{}

\begin{frame}[t]
  \frametitle{Deployment}

  \begin{enumerate}
  \item On startup, a node begins listening for a deployment broadcast
  \item The sink node has the option to begin broadcasting the deployment message (300ms)
    \begin{itemize}
    \item[--] The maximum number of nodes and deployment type are held in the deployment packet
    \end{itemize}
  \item A new node is turned on as is moves away from the broadcasting node, it runs the specified deployment test
  \item The node then sends the stop signal once deployed and begins deploying if it is not the last node
  \item If the node is the final node, it will send the stop signal and begin transmitting data to the sink node(250ms)
    
  \end{enumerate}
  
\end{frame}

\begin{frame}[t]
  \frametitle{Deployment tests}

  RSSI test
  \begin{itemize}
  \item Records how many of the previous 16 packets were below the RSSI threshold
  \item Compares the number of low RSSI values to a threshold (10)
  \end{itemize}

  \vfill

  Packet Loss test
  \begin{itemize}
  \item Records how many of the previous 15 packets were lost, based on sequence number
  \item Compares the number of lost packets to a threshold (4)
  \end{itemize}
\end{frame}


\begin{frame}
  \frametitle{Data Consistency}

  \begin{itemize}
  \item Messages wait for an ACK before sending the next message when streaming data
  \item PINGs and PONGs
  \item An acknowledgment is also used to signal that the STOP signal has been received
  \item Information about total packet loss is relayed with the packet at each hop
  \end{itemize}
  
\end{frame}
