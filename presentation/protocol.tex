\section{Protocol}
\subsection*{}


\begin{frame}[t]
  \frametitle{Characterizations of Ad Hoc Networks}
  
  Lack of fixed network infrastructure
  \begin{itemize}
  \item Dedicated routers absent
  \end{itemize}

  \vfill
  
  Mobility
  \begin{itemize}
  \item Hosts in network may be mobile
  \item In our implementation: only mobile until deployed
  \end{itemize}

  \vfill

  Shared Channel
  \begin{itemize}
  \item Flows in vicinity contend with each other
  \item Parts of flows traversing multiple hops will contend with itself
  \end{itemize}

  \vfill

  Limited Bandwidth
  \begin{itemize}
  \item Possibly only a few kilobytes per second
  \end{itemize}

  \vfill

  \begin{flushleft}
    \begin{tiny}
      \begin{minipage}{1.0\linewidth}
        \fullcite{AK01}
      \end{minipage}
    \end{tiny}
  \end{flushleft}
  
\end{frame}

\begin{frame}[t]
  \frametitle{TCP vs ATP Protocol\dots}

  TCP (Transmission Control Protocol)
  \begin{itemize}
  \item Window based transmissions
  \item Slow start
  \item Dependant on ACKs
  \end{itemize}

  \vfill

  ATP (Ad Hoc Transport Protocol)
  \begin{itemize}
  \item Rate based transmissions
  \item Quick start
  \item Use of SACKs
  \end{itemize}

  \vfill

  \begin{flushleft}
    \begin{tiny}
      \begin{minipage}{1.0\linewidth}
        \fullcite{AK01}
      \end{minipage}
    \end{tiny}
  \end{flushleft}
  
\end{frame}

\begin{frame}[t]
  \frametitle{Window Based vs Rate BAsed Transmissions}

  TCP
  \begin{itemize}
  \item Control data transmission by adjusting the congestion window size based on the ACK’s received
  \item Send two packets for each ACK received
    \begin{itemize}
      \item[--] Additive Increase, Multiplicative Decrease
      \item[--] Leads to burstiness of data
    \end{itemize}
  \end{itemize}

  \vfill

  ATP
  \begin{itemize}
  \item Rate-based mechanisms control the transmission rate based on the measurement taken at the end host
  \item No burstiness
  \item ACK bunching happens
    \begin{itemize}
      \item[--] Several acks received at the same time
      \item[--] Leads to flooding of network with window based transmissions
    \end{itemize}
  \end{itemize}

    \vfill

  \begin{flushleft}
    \begin{tiny}
      \begin{minipage}{1.0\linewidth}
        \fullcite{AK01}
      \end{minipage}
    \end{tiny}
  \end{flushleft}

\end{frame}    

\begin{frame}[t]
  \frametitle{Slow vs. Quick Start}

  TCP
  \begin{itemize}
    \item Starts off slow, and gradually increases
    \item Probes the network for available bandwidth
    \item May take a while to get to true available bandwidth
    \item Loss occurs? return to slow start phase
  \end{itemize}

  \vfill
  
  ATP
  \begin{itemize}
  \item Starts off quickly and may decrease
    \begin{itemize}
    \item[--]Packet loss is more frequent so a connection may spend most of its life in the slow-start phase if using TCP
    \end{itemize}
  \end{itemize}
  
    \vfill

  \begin{flushleft}
    \begin{tiny}
      \begin{minipage}{1.0\linewidth}
        \fullcite{AK01}
      \end{minipage}
    \end{tiny}
  \end{flushleft}
  
\end{frame}

\begin{frame}[t]
  \frametitle{Decoupling of Congestion Control and Reliability}

  In TCP, congestion control and reliability are tightly coupled through dependence on ACK arrival

  \vfill
  
  With ATP you have\dots

  \begin{itemize}
  \item Feedback about the strength of the connection is “piggybacked” onto packets being routed to their destination
  \item Selective ACK’s used to report losses of connections, or holes, in the data stream
    \begin{itemize}
    \item[--] According to RFC 2018, a selective ACK, or SACK, is sent to inform a sender about all segments that have arrived successfully.
    \end{itemize}
  \end{itemize}

  \vfill

  \begin{flushleft}
    \begin{tiny}
      \begin{minipage}{1.0\linewidth}
        \fullcite{AK01}
      \end{minipage}
    \end{tiny}
  \end{flushleft}
  
\end{frame}
    
\begin{frame}[t]
  \frametitle{In Relation to Our Implementation}

  \vfill
  
  We also “piggyback” statistical information about the strength of the connection on each message sent to the sink
  \vfill
  
  For future work, SACK’s could be used exclusively in our network

  \vspace{\vfill}
  
  \begin{flushleft}
    \begin{tiny}
      \begin{minipage}{1.0\linewidth}
        \fullcite{AK01}
      \end{minipage}
    \end{tiny}
  \end{flushleft}

\end{frame}
